\chapter{Wnioski}
W ramach niniejszej pracy magisterskiej opracowany został system sterowania ruchem drogowym wraz z algorytmem sterowania ruchem. Została również przygotowana platforma do testowania algorytmu sterowania ruchem z symulatorem ruchu drogowego opartym o automat komórkowy.

Przygotowany algorytm porównany został z ruchem drogowym przy braku sterowania jak i ze sterowaniem prostym, stałoczasowym, programem sygnalizacji. W wyniku wykonanych badań wykazane zostały wady i zalety opracowanego algorytmu.

Podstawową zaletą jest zwiększenie płynności ruchu w porównaniu z pozostałymi badanymi przypadkami, co pokazane jest przez mniejszą średnią liczbę zatrzymań pojazdów. Zmniejszona średnia prędkość pojazdów, w porównaniu z brakiem sterowania, rekompensowana jest przez zwiększone bezpieczeństwo ruchu, wynikające bezpośrednio z zastsowania sygnalizacji świetlnej.

Wadami opracowanego systemu są ograniczenia wynikające z metody ustalenia które sygnalizatory powinny w danej chwili zezwalać na przejazd. Wynikające problemy są dobrym punktem wyjściowym do dalszego rozwoju opracowanego algorytmu.

Przykładem możliwego rozwoju jest rozwiązanie problemu faworyzacji stanów sygnalizacji o większej liczbie strumieni ruchu, czy faworyzacja strumieni nadjeżdżających z sąsiadujących obszarów kontrolowanych opracowanym algorytmem.