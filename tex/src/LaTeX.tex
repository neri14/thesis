\nonstopmode

%\documentclass[twoside]{pwrthesis}
\documentclass[oneside]{iisthesis}
% ---
%\usepackage[MeX]{polski}
%\usepackage[polish]{babel}
%\usepackage[cp1250]{inputenc}

\usepackage{polski}
\usepackage[polish]{babel}
\usepackage[utf8]{inputenc}
\usepackage{graphicx}

% Dodane przeze mnie d
\usepackage{fancyvrb} % dla srodowiska Verbatim
\usepackage{color}
\usepackage{lscape}
\usepackage{amssymb}

\usepackage{icomma}

% definicje kolorow
\definecolor{ciemnoSzary}{rgb}{0.15,0.15,0.15}
\definecolor{szary}{rgb}{0.5,0.5,0.5}
\definecolor{jasnoSzary}{rgb}{0.2,0.2,0.2}

% Konfiguracja verbatima
\fvset{
	frame=single,
	numbers=left,
	fontsize=\footnotesize,
	numbersep=12pt,
%	framerule=.5mm,
	rulecolor=\color{ciemnoSzary},
%	fillcolor=\color{jasnoSzary},
	framesep=4pt,
	stepnumber=1,
	numberblanklines=false,
	tabsize=2,
%	formatcom=\color{szary}
}

\begin{document}

\title{Opracowanie inteligentnego systemu sterowania ruchem drogowym.}
\author{inż. Przemysław Rokosz}
\advisor{dr inż. Grzegorz Filcek}
\instituteLogo{logos/pwr}
\slowaKluczowe{pierwsze\\drugie\\trzecie}

\date{\number\the\year}

% Wstawienie abstractu pracy
	%\input {abstract}
	
\abstractSH{
Bardzo krótkie streszczenie w którym powinno się znaleźć omówienie tematu pracy i poruszanych terminów. Tekst ten nie może być zbyt długi. }

\abstractPL{
	AbstraktPL
}
\abstractEN{
	AbstraktEN
}

\maketitle
\textpages

%\chapter*{Wprowadzenie}

%chapter -> section -> subsection

\chapter{Wstęp}
\section{Wprowadzenie}
wprowadzenie w tematykę pracy
\section{Cel pracy}
cel pracy, motywacja
\section{Układ pracy}
krótki opis co w rozdziałach - możliwe że większość znajduje się w przeglądzie literatury
\section{Przegląd literatury}
\subsection{Wymagania prawne dotyczące sygnalizacji świetlnej}
Podstawowe zasady projektowania sygnalizacji świetlnej oraz jej programów opisuje Rozporządzenie Ministra Infrastruktury z dnia 3 lipca 2003 roku \cite{rozporzadzenie}.
Ze względu na tematykę pracy, najważniejszy jest załącznik nr 3 rozporządzenia opisujący szczegółowe warunki techniczne dla sygnałów drogowych w tym wymagania dotyczące programu sygnalizacji świetlnej w punkcie ósmym.
Wymagania programu sygnalizacji świetlnej zaczynają się od zasad ogólnych, opisujących między innymi programy przejściowe z i do sygnału ostrzegawczego (oznaczającego wyłączenie sygnalizacji). Następnie opisane są wymagania formalne w tym wymagania czasowe dotyczące sygnałów, opisane dokładniej w rozdziale \ref{sec:model_ograniczenia}. na stronie \pageref{sec:model_ograniczenia}.
W dalszej części opisane są wymagania bezpieczeństwa ruchu w tym metoda wyliczenia długości czasów międzyzielonych, czyli odstęp czasu służący zabezpieczeniu aby pojazdy w kolizyjnych strumieniach ruchu nie znalazły się w tym samym miejscu i czasie. Metoda wyliczania czasów międzyzielonych również opisana jest w rozdziale \ref{sec:model_ograniczenia}.

\subsection{Metody stosowane w sterowaniu ruchem drogowym}
Usystematyzowany przegląd metod i algorytmów sterowania ruchem drogowym przedstawiają w swojej pracy Piotr Kawalec z Politechniki Warszawskiej oraz Sylwia Sobieszuk-Durka z Urzędu m. st. Warszawy \cite{kawalec+sobieszuk-durka}. Dzielą oni metody sterowania ruchem drogowym przede wszystkim na optymalizujące i nieoptymalizujące funkcji celu. Metody optymalizujące funkcję celu mogą być przeniesione na teorię systemów sterowania oraz pozwalają na ocenę opracowanego systemu sterowania ruchem drogowym.

Autorzy zauważają, że nawet najlepszy algorytm sterowania ruchem nie da dobrych efektów jeśli zostanie zastosowany dla pojedynczego skrzyżowania. Stąd konieczne jest opracowywanie systemów sterowania z myślą o większych zespołach w których sygnał sterujący wpływa na wiele miejsc w zakresie rozpatrywanego obszaru.

We wspomnianej pracy przedstawione zostały przykładowe struktury systemów adaptacyjnego sterowania ruchem drogowym co jest dobrym punktem wyjścia do projektowania inteligentnego systemu sterowania ruchem. Opis opracowanej struktury sterowania znajduje się w rozdziale \ref{sec:model_opis} na stronie \pageref{sec:model_opis}.

TODO praca \cite{ruchaj}

\subsection{Rozwiązania w symulacji ruchu drogowego}

referencja \ref{chap:symulacja}

TODO praca \cite{nasch}

TODO praca \cite{bernas+placzek}

TODO praca \cite{bartodziej}

\chapter{Przegląd literatury}
\section{Wymagania prawne dotyczące sygnalizacji świetlnej}
Podstawowe zasady projektowania sygnalizacji świetlnej oraz jej programów opisuje Rozporządzenie Ministra Infrastruktury z dnia 3 lipca 2003 roku \cite{rozporzadzenie}.
Ze względu na tematykę pracy, najważniejszy jest załącznik nr 3 rozporządzenia opisujący szczegółowe warunki techniczne dla sygnałów drogowych w tym wymagania dotyczące programu sygnalizacji świetlnej w punkcie ósmym.
Wymagania programu sygnalizacji zaczynają się od zasad ogólnych, opisujących między innymi programy przejściowe z i do sygnału ostrzegawczego (oznaczającego wyłączenie sygnalizacji). Następnie opisane są wymagania formalne w tym wymagania czasowe dotyczące sygnałów, opisane dokładniej w rozdziale \ref{sec:model_ograniczenia}. na stronie \pageref{sec:model_ograniczenia}.
W dalszej części opisane są wymagania bezpieczeństwa ruchu w tym metoda wyliczenia długości czasów międzyzielonych, czyli odstęp czasu służący zabezpieczeniu aby pojazdy poruszające się w kolizyjnych strumieniach ruchu nie znalazły się w tym samym miejscu i czasie. Metoda wyliczania czasów międzyzielonych również opisana jest w rozdziale \ref{sec:model_ograniczenia}.

\section{Metody stosowane w sterowaniu ruchem drogowym}
Usystematyzowany przegląd metod i algorytmów sterowania ruchem drogowym przedstawiają w swojej pracy Piotr Kawalec z Politechniki Warszawskiej oraz Sylwia Sobieszuk\-Durka z Urzędu m. st. Warszawy \cite{kawalec+sobieszuk-durka}. Dzielą oni metody sterowania ruchem drogowym na optymalizujące i nieoptymalizujące funkcji celu.
Przedstawione wartości optymalizowane mogą być wprost zastosowane do oceny opracowywanego systemu sterowania ruchem drogowym.

Autorzy zauważają, że nawet najlepszy algorytm sterowania ruchem nie da dobrych efektów jeśli zostanie zastosowany dla pojedynczego skrzyżowania. Stąd konieczne jest opracowanie systemów sterowania z myślą o większych zespołach w których sygnał sterujący wpływa na wiele miejsc w zakresie rozpatrywanego obszaru.

We wspomnianej pracy przedstawione zostały przykładowe struktury systemów adaptacyjnego sterowania ruchem drogowym co jest dobrym punktem wyjścia do projektowania inteligentnego systemu sterowania ruchem. Opracowana w ramach tej pracy struktura sterowania opisana została w rozdziale \ref{sec:model_opis} na stronie \pageref{sec:model_opis}.

W swojej rozprawie doktorskiej \cite{ruchaj} Marcin Ruchaj przytacza algorytmy sterowania ruchem drogowym z podziałem na sterowanie stałoczasowe - ze stałym programem sygnalizacji, zmiennoczasowe - adaptujące się do warunków ruchu, oraz sterowanie wykorzystujące logikę rozmytą czy metody sztucznej inteligencji, przedstawione również w pracy Tahere Royani Javad Haddadni i Mohammada Alipoora \cite{royani+haddadnia+alipoor}. W swoim referacie proponują oni zastosowanie rozmytych sieci neuronowych do sterowania ruchem i algorytmu genetycznego w celu regulacji parametrów pracy sieci neuronowej.

\section{Rozwiązania w symulacji ruchu drogowego}
Dla celów symulacji ruchu drogowego często wykorzystywane są automaty komórkowe. Zaletą zastosowania automatów komórkowych jest ich oparcie o proste zasady i wysoka wydajność obliczeniowa. Popularny model symulacji ruchu pojazdów opracowali Kai Nagel i Michael Schreckenberg w roku 1992 \cite{nasch}. We wspomnianej pracy opisują oni automat komórkowy którego działanie opiera się na 4 etapach: przyśpieszeniu, hamowaniu, losowości i przesunięciu. Zastosowanie automatów komórkowych w symulacji różnych sytuacji ruchu drogowego opisuje w swojej pracy dyplomowej Maciej Bartodziej \cite{bartodziej}.

Ze względu na swoją prostotę automaty komórkowe można również prosto modyfikować. Przykład symulacji ruchu drogowego w zmiennych warunkach pogodowych przedstawiają w swoim artykule Marcin Bernaś i Bartłomiej Płaczek \cite{bernas+placzek}. Dla celu niniejszej pracy opracowany został symulator stosujący automat komórkowy zmodyfikowany w celu umożliwienia symulacji zespołów skrzyżowań, dokładny opis automatu i jego modyfikacji znajduje się w rozdziale \ref{chap:symulacja} na stronie \pageref{chap:symulacja}.

\section{Istniejące systemy sterowania ruchem}
Inteligentne systemy transportu (ITS\footnote{intelligent transportation systems}) zyskują coraz większą popularność w zarządzaniu ruchem drogowym.
Mają one na celu skrócenie czasu podróży, poprawę bezpieczeństwa czy zwiększenie komfortu podróży co w swojej prezentacji, przedstawiające innowacje w zarządzaniu transportem miejeskim zauważają Aneta Pluta-Zarembska, Marzenna Cichosz i Katarzyna Nowicka ze Szkoły Głównej Handlowej w Warszawie \cite{pluta-zaremba+cichosz+nowicka}. Jako przykład wdrożenia ITS przedstawiony został inteligentny system zarządzania transportem we Wrocławiu. Cel Wrocławskiego systemu, przyspieszenie ruchu o 20\%,  ma zostać osiągnięty, do III kwartału 2014 roku, dzięki zastosowaniu sterowania ruchem na 153 skrzyżowaniach, usprawnieniu komunikacji miejskiej czy dynamicznych informacjach dla kierowców.

Podobny system, zastosowany w trójmieście, prezentują, w artykule na temat systemu TRISTAR, Kazimierz Jamroz i Jacek Oskarbski \cite{jamroz+oskarbski}. Jak zauważają, motywacją wprowadzenia systemu jest między innymi: zatłoczenie infrastruktury drogowej, ryzyko zdarzeń drogowych czy brak informacji o warunkach ruchu. System TRISTAR składa się z połączonych komponentów, takich jak:
\begin{itemize}
	\item zarządzanie ruchem drogowym
	\item zarządzanie transportem zbiorowym
	\item zarządzanie służbami ratowniczymi
	\item zintegrowany system informacji
	\item zarządzanie transportem towarowym
\end{itemize}
Zarządzenie ruchem drogowym obejmuje zarządzanie ruchem miejskim jak i ruchem na okolicznych drogach krajowych i szybkiego ruchu (w postaci trójmiejskiej obwodnicy).

Kolejnym przykładem systemu wdrażanego w Polsce jest podhalański system sterowania ruchem, przedstawiony w artykule Patryka Zakrzewskiego dla magazynu \textit{Drogi Budownictwo Infrastrukturalne} \cite{zakrzewski}. Podobnie jak w przypadku poprzednich dwóch systemów, celami systemu jest zwiększenie bezpieczeństwa i zmniejszenie zatłoczenia na drogach. Jednakże, w przeciwieństwie do nich, podhalański system projektowany był z myślą o zastosowaniu poza warunkami miejskimi. Jego głównym elementem jest monitoring dróg regionu oraz przedstawienie kierowcom informacji o czasie przejazdu do miejscowości orientacyjnych.

\chapter{Model sterowania ruchem drogowym}
\label{chap:model}
Przyjęty model sterowania zakłada, że system sterowania ruchem drogowym jest systemem dyskretnym. Podobnie, przyjęty model symulatora w postaci automatu komórkowego zakłada symulację ruchu w postaci dyskretnych stanów. Długość każdego stanu to jedna sekunda co daje jedną sekundę na wyznaczneie nowego stanu sygnalizatorów na podstawie akutalnego stanu czujników.

Obszar, obiekt, sterowania udostępnia dane kontrolerom w postaci wartości dwóch typów czujników. Czujnika przepływu pojazdów na dojeździe do sygnalizatora, w którym przepływ wyznaczany jest na podstawie liczby pojazdów mijających czujnik w zadanym czasie, oraz czujnika kolejki. Długość kolejki wyznaczana jest jako liczba pojazdów znajdujących się na wybranym odcinku drogi w danej chwili czasu.

Kontroler wpływa na sterowany obszar przez zmianę stanu sygnalizatorów. Pojazdy reagują na stan sygnalizatora mijając go jedynie jeśli zezwala on na przejazd.

\section{Model sterowania}
\label{sec:model_opis}

Na rysunku \ref{fig:model} zaprezentowany został ogólny model sterowania ruchem drogowym.
Zespół skrzyżowań objęty sterowaniem podzielony jest na obszary.
Każdy obszar obejmuje pojedyncze skrzyżowanie lub jego autonomiczną część, czyli taką w której dojazd strumieni ruchu do miejsca przecięcia jest kontrolowany bezpośrednio jedynie przez sygnalizatory znajdujące się w danej części skrzyżowania.

Każdy obszar sterowany jest przez pojedynczy kontroler.
Kontroler jako dane wejściowe przyjmuje stan systemu w poprzedniej chwili czasu,
zawierający wielkości mierzone przez czujniki jak i sterowania wszystkich kontrolerów. Zapewnia to możliwość współpracy sąsiadujących kontrolerów.

Powyższy, uproszczony, model ograniczony jest do sterowania jednym typem pojzadów. Nie uwzględnia on również sterowania ruchem pieszych i rowerzystów na przjściach dla pieszych oraz przejazdach dla rowerów. W tak zdefiniowanym modelu można przyjąć, że sygnalizatory dla pieszych i rowerzystów, nadają sygnał zezwalający (zielony) w czasie gdy przecinające potoki ruchu otrzymują sygnał czerwony.
Aby zapewnić że sytuacja tego typu jest możliwa należy wymusić przynajmniej jednorazowe, dla każdego sygnalizatora, użycie sygnału czerwonego, zgodnie z ograniczeniami opisanymi w sekcji \ref{sec:model_ograniczenia}.

\begin{figure}[h]
    \centering
    \includegraphics[width=0.8\textwidth]{images/model.pdf}
    \caption{Model systemu sterowania}
    \label{fig:model}
\end{figure}

\begin{equation}
	\begin{array}{c}
		U_i (n) = \left[ u_{i, j} (n) \right]_{j \in <1,s>}\\
		i \in <1,o>\\
		u_{i, j} (n) \in \left\{ \textrm{CZERWONY, CZERWONO-ZOLTY, ZIELONY, ZOLTY} \right\}\\
		o \in \mathbb{N}\\
		s \in \mathbb{N}
	\end{array}
\end{equation}

u_{i,j} (n) \textrm{ -- stan j-tego sygnalizatora w i-tym obszarze w chwili n}

o -- liczba obszarów sterowania

s -- liczba sygnalizatorów w obszarze

\begin{equation}
	\begin{array}{c}
		Z_i (n) = \left[ z_{i, j} (n) \right]_{j \in <1,r>}\\
		i \in <1,o>\\
		z_{i, j} (n) \in \mathbb{N}\\
		r \in \mathbb{N}
	\end{array}
\end{equation}

z_{i,j} (n) \textrm{ -- przepływ pojazdów z kierunku j w i-tym obszarze w chwili n}

r -- liczba wlotów do danego obszaru sterowania

\begin{equation}
	\begin{array}{c}
		X_i (n) = \left[
			\begin{array}{c}
				x_{i, j, 1} (n) \\ x_{i, j, 2} (n) \\ x_{i, j, 3} (n)
			\end{array}
		\right]_{j \in <1,s>}\\
		i \in <1,o>\\
		x_{i, j, 1} (n) \in \mathbb{N}\\
		x_{i, j, 2} (n) \in \mathbb{N}\\
		x_{i, j, 3} (n) \in \left\{ \textrm{CZERWONY, CZERWONO-ZOLTY, ZIELONY, ZOLTY} \right\}
	\end{array}
\end{equation}

x_{i, j, 1} (n) \textrm{ -- przepływ na dojeździe do j-tego sygnalizatora w i-tym obszarze}

x_{i, j, 2} (n) \textrm{ -- kolejka przed j-tym sygnalizatorem w i-tym obszarze}

x_{i, j, 3} (n) \textrm{ -- stan j-tego sygnalizatora w i-tym obszarze}

\vspace{0.5cm}
Przepływy pojazdów mierzone są w pojazdach na godzinę, stany kolejek mierzone są w postaci liczby pojazdów.

\section{Wymagania sterowania ruchem drogowym}
\label{sec:model_ograniczenia}
Ograniczenia dotyczące sterowania ruchem drogowym zostały ustalone w rozporządzeniu ministra infrastruktury \cite{rozporzadzenie}. Wymagania te można sprowadzić do zestawu wymagań formalnych i wymagań bezpieczeństwa:
\subsection{Wymagania formalne}
\begin{itemize}
	\item Sekwencja sygnałów czerwony, czerwono-żółty\footnote{sygnał czerwony z żółtym, przygotowanie do jazdy}, zielony, żółty, czerwony
	\item Długość sygnału żółtego powinna wynosić 3 sekundy
	\item Długość sygnału czerwono-żółtego powinna wynosić 1 sekundę
	\item Dla sygnalizacji akomodacyjnej/acyklicznej minimalna długość sygnału zielonego to 5 sekund
\end{itemize}

\subsection{Wymagania bezpieczeństwa}
Rozporządzenie definiuje podział par strumieni ruchu na strumienie niekolizyjne, strumienie kolizyjne o dopuszczalnym jednoczesnym zezwoleniu na ruch oraz strumienie kolizyjne o niedopuszczalnym jednoczesnym zezwoleniu na ruch. Ze względów bezpieczeństwa, na potrzeby opracowywanego systemu przyjęto, że strumienie kolizyjne nigdy nie mogą otrzymać jednoczesnego zezwolenia na ruch.

Zdefiniowano metodę obliczania czasów międzyzielonych dla kolizyjnych par strumieni. Zapewniają one minimalny czas w którym strumień ewakuujący się zdąży minąć punkt kolizji zanim osiągnie go strumień dojeżdżający. Minimalny czas międzyzielony wyznacza się na podstawie czasu trwania sygnału żółtego dla strumienia ewakuującego się, czasu ewakuacji strumienia ewakuującego się oraz czasu dojazdu strumienia dojeżdżającego.

\begin{equation}
	t^{min}_{m} (i,j) = t_{z} + t_{e} (i,j) - t_{d} (i,j)
\end{equation}

t^{min}_{m} (i,j) \textrm{ -- minimalny czas międzyzielony dla pary strumieni (i,j) [s]}

t_{z} \textrm{ -- długość sygnału żółtego dla strumienia ewakuującego się [s]}

t_{e} (i,j) \textrm{ -- czas ewakuacji strumienia i poza punkt kolizji ze strumieniem j [s]}

t_{d} (i,j) \textrm{ -- czas dojazdu strumienia j do punktu kolizji ze strumieniem i [s]}

t^{min}_{m} (i,j) = 0 \textrm{ jeśli obliczona wartość jest mniejsza od 0}

\begin{equation}
	t_{e} (i,j) = \frac{s_{e} (i,j) + I_p}{v_{e} (i)}
\end{equation}

s_{e} (i,j) \textrm{ -- droga strumienia ewakuującego się od linii zatrzymania do punktu kolizji[m]}

I_p \textrm{ -- wartość wydłużająca drogę ewakuacji, dla strumienia pojazdów 10 metrów}

v_{e} (i) \textrm{ -- prędkość ewakuacji [m/s], prędkość dopuszczalna na wlocie, nie większa niż 14 m/s}

\begin{equation}
	t_{d} (i,j) = \frac{s_{d} (i,j)}{v_{d} (j)} + 1
\end{equation}

s_{d} (i,j) \textrm{ -- droga strumienia dojeżdżającego od linii zatrzymania do punktu kolizji [m]}

v_{d} (i) \textrm{ -- prędkość ewakuacji [m/s], prędkość dopuszczalna na wlocie}

\chapter{Algorytm sterowania ruchem drogowym}
TODO jak sterujemy (na podstawie jakich kryteriów oceny itp), jak oceniamy skuteczność algorytmu
opis oceny na podstawie listy w rozdziale \ref{chap:ocena}

uwzględnić: - światło nie może być zawsze czerwone ani zawsze zielone

\chapter{Opracowany system sterowania ruchem}
System, przygotowany do badania algorytmu sterowania, składa się z trzech podstawowych komponentów:
\begin{description}
	\item[symulator] --
		symulujący ruch drogowy, udostępniający dane z czujników i przyjmujący ustawienia sygnalizatorów.
		Jest on również źródłem czasu co pozwala na synchronizację działania całego systemu. Opisany w sekcji \ref{chap:symulacja}.
	\item[kontroler] --
		osobny dla każdego obszaru, otrzymuje dane z symulatora i wyznacza sterowania sygnalizatorów.
		Może komunikować się z innymi kontrolerami w celu wymiany danych o sąsiednich obszarach. Więcej w sekcji \ref{chap:kontroler}.
	\item[serwer] --
		zapewnia abstrakcję komunikacji. Pozwala na wymiane wiadomości w postaci zdarzeń.
		Działanie komunikacji w systemie opisane jest w sekcji \ref{chap:komunikacja}.
\end{description}

\section{Technologia}
System sterowania został przygotowany w języku C++ w standardzie C++03 z wykorzystaniem biblioteki Boost 1.55.0 \cite{boost}. Do komunikacji sieciowej wykorzystano bibliotekę Google Protocol Buffer \cite{protobuf}. Przygotowane testy jednostkowe wykorzystują bibliotekę Google Test \cite{gtest}.

Obszar symulowany wczytywany jest z plików XML z opisem. Pliki z opisem zawierają również obliczone czasy międzyzielone pomiędzy kolizyjnymi strumieniami ruchu.

\section{Obszar badania algorytmu}
Dla celów badania algorytmu przygotowany został opis obszaru okolic placu Grunwaldzkiego we Wrocławiu, przedstawionego na rysunku \ref{fig:mapa_czysta}.

\begin{figure}[h]
    \centering
    \includegraphics[width=1.0\textwidth]{images/mapa_czysta.png}
    \caption{Mapa obszaru badania algorytmu, przygotowana na podstawie Google Maps \cite{google_maps}}
    \label{fig:mapa_czysta}
\end{figure}


TODO opis zespołu skrzyżowań na którym prowadzone są testy algorymu

opis o podziale na obszary, może zmienić mapę na schematyczną mapę z sygnalizatorami

informacja że dane ruchu nie są rzeczywiste bo symulator i tak jest przybliżony

\section{Symulator}
\label{chap:symulacja}
TODO jak symulujemy - o modelu Nagela-Schreckenberga (NaSch) i jego modyfikacjach

\section{Kontroler}
\label{chap:kontroler}
TODO opis kontrolera

\section{Komunikacja}
\label{chap:komunikacja}
TODO opis komunikacji

\chapter{Ocena algorytmu}
ocena działania algorytmu na podstawie badań

\chapter{Wnioski}
W ramach niniejszej pracy magisterskiej opracowany został system sterowania ruchem drogowym wraz z algorytmem sterowania ruchem. Została również przygotowana platforma do testowania algorytmu sterowania ruchem z symulatorem ruchu drogowego opartym o automat komórkowy.

Przygotowany algorytm porównany został z ruchem drogowym przy braku sterowania jak i ze sterowaniem prostym, stałoczasowym, programem sygnalizacji. W wyniku wykonanych badań wykazane zostały wady i zalety opracowanego algorytmu.

Podstawową zaletą jest zwiększenie płynności ruchu w porównaniu z pozostałymi badanymi przypadkami, co pokazane jest przez mniejszą średnią liczbę zatrzymań pojazdów. Zmniejszona średnia prędkość pojazdów, w porównaniu z brakiem sterowania, rekompensowana jest przez zwiększone bezpieczeństwo ruchu, wynikające bezpośrednio z zastsowania sygnalizacji świetlnej.

Wadami opracowanego systemu są ograniczenia wynikające z metody ustalenia które sygnalizatory powinny w danej chwili zezwalać na przejazd. Wynikające problemy są dobrym punktem wyjściowym do dalszego rozwoju opracowanego algorytmu.

Przykładem możliwego rozwoju jest rozwiązanie problemu faworyzacji stanów sygnalizacji o większej liczbie strumieni ruchu, czy faworyzacja strumieni nadjeżdżających z sąsiadujących obszarów kontrolowanych opracowanym algorytmem.

\chapter{Dummy Rozdział}
Lorem ipsum dolor sit amet, consectetur adipiscing elit. Etiam fermentum hendrerit risus nec ultrices. Sed adipiscing adipiscing nunc, non convallis mauris facilisis nec. Nunc lorem elit, viverra vitae neque et, commodo egestas enim. Proin a ante non leo consequat lobortis hendrerit ut nisl. Phasellus quis feugiat urna, sit amet rutrum sem. Pellentesque habitant morbi tristique senectus et netus et malesuada fames ac turpis egestas. Donec a mattis nulla, et tincidunt purus. Etiam et sagittis tortor, ut venenatis eros. Nullam vulputate scelerisque molestie. Sed consequat ipsum et gravida viverra. Phasellus nec congue augue. Aenean pretium erat elit, eget pretium erat vulputate ut. Pellentesque porta imperdiet ante, vel pellentesque est lobortis convallis. Curabitur at augue tincidunt velit dictum pellentesque sit amet in massa. Dane znajdują się w tabeli ~\ref{tabela:testowa} na stronie ~\pageref{tabela:testowa}.
\begin{table}[h] %h - here - nie do końca ale działa
	\caption{A normal caption}
	\begin{tabular}{ r|c|c| }
		\multicolumn{1}{r}{}
 		&  \multicolumn{1}{c}{noninteractive}
 		& \multicolumn{1}{c}{interactive} \\
		\cline{2-3}
		massively multiple & Library & University \\
		\cline{2-3}
		one-to-one & Book & Tutor \\
		\cline{2-3}
	\end{tabular}
	\label{tabela:testowa}
\end{table}
test \cite{abramowitz+stegun}

\pagestyle{plain}
%\pagenumbering{gobble}

\listoffigures
\listoftables

\bibliographystyle{iisthesis}
\bibliography{biblio}

\appendix
\chapter{Wykorzystane pojęcia}

\begin{description}

\item[autonomiczna część skrzyżowania] --
część skrzyżowania, w której dojazd potoków ruchu do miejsca przecięcia jest kontrolowany
jedynie przez sygnalizatory znajdujące się w danej części skrzyżowania

\item[chwila czasu] -- TODO

\item[czas międzyzielony] -- TODO

\item[czujnik] -- TODO

\item[kontroler] -- TODO

\item[obszar sterowania] --
obszar, obejmujący pojedyncze skrzyżowanie lub autonomiczną część skrzyżowania,
kontrolowany przez pojedynczy kontroler

\end{description}

\chapter{Zawartość załączonej płyty}

Załączona płyta zawiera kopię repozytorium pracy magisterskiej.

\begin{description}

\item[katalog 3rdparty] --
biblioteki potrzebne do skompilowania przygotowanego systemu

\item[katalog tex] --
tekst pracy magisterskiej w formacie \LaTeX

\item[katalog thesis] --
kod źródłowy przygotowanego systemu


\end{description}

\end{document}

