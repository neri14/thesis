\chapter{Przegląd literatury}
\section{Wymagania prawne dotyczące sygnalizacji świetlnej}
Podstawowe zasady projektowania sygnalizacji świetlnej oraz jej programów opisuje Rozporządzenie Ministra Infrastruktury z dnia 3 lipca 2003 roku \cite{rozporzadzenie}.
Ze względu na tematykę pracy, najważniejszy jest załącznik nr 3 rozporządzenia opisujący szczegółowe warunki techniczne dla sygnałów drogowych w tym wymagania dotyczące programu sygnalizacji świetlnej w punkcie ósmym.
Wymagania programu sygnalizacji zaczynają się od zasad ogólnych, opisujących między innymi programy przejściowe z i do sygnału ostrzegawczego (oznaczającego wyłączenie sygnalizacji). Następnie opisane są wymagania formalne w tym wymagania czasowe dotyczące sygnałów, opisane dokładniej w rozdziale \ref{sec:model_ograniczenia}. na stronie \pageref{sec:model_ograniczenia}.
W dalszej części opisane są wymagania bezpieczeństwa ruchu w tym metoda wyliczenia długości czasów międzyzielonych, czyli odstęp czasu służący zabezpieczeniu aby pojazdy poruszające się w kolizyjnych strumieniach ruchu nie znalazły się w tym samym miejscu i czasie. Metoda wyliczania czasów międzyzielonych również opisana jest w rozdziale \ref{sec:model_ograniczenia}.

\section{Metody stosowane w sterowaniu ruchem drogowym}
Usystematyzowany przegląd metod i algorytmów sterowania ruchem drogowym przedstawiają w swojej pracy Piotr Kawalec z Politechniki Warszawskiej oraz Sylwia Sobieszuk\-Durka z Urzędu m. st. Warszawy \cite{kawalec+sobieszuk-durka}. Dzielą oni metody sterowania ruchem drogowym na optymalizujące i nieoptymalizujące funkcji celu.
Przedstawione wartości optymalizowane mogą być wprost zastosowane do oceny opracowywanego systemu sterowania ruchem drogowym.

Autorzy zauważają, że nawet najlepszy algorytm sterowania ruchem nie da dobrych efektów jeśli zostanie zastosowany dla pojedynczego skrzyżowania. Stąd konieczne jest opracowanie systemów sterowania z myślą o większych zespołach w których sygnał sterujący wpływa na wiele miejsc w zakresie rozpatrywanego obszaru.

We wspomnianej pracy przedstawione zostały przykładowe struktury systemów adaptacyjnego sterowania ruchem drogowym co jest dobrym punktem wyjścia do projektowania inteligentnego systemu sterowania ruchem. Opracowana w ramach tej pracy struktura sterowania opisana została w rozdziale \ref{sec:model_opis} na stronie \pageref{sec:model_opis}.

W swojej rozprawie doktorskiej \cite{ruchaj} Marcin Ruchaj przytacza algorytmy sterowania ruchem drogowym z podziałem na sterowanie stałoczasowe - ze stałym programem sygnalizacji, zmiennoczasowe - adaptujące się do warunków ruchu, oraz sterowanie wykorzystujące logikę rozmytą czy metody sztucznej inteligencji, przedstawione również w pracy Tahere Royani Javad Haddadni i Mohammada Alipoora \cite{royani+haddadnia+alipoor}. W swoim referacie proponują oni zastosowanie rozmytych sieci neuronowych do sterowania ruchem i algorytmu genetycznego w celu regulacji parametrów pracy sieci neuronowej.

\section{Rozwiązania w symulacji ruchu drogowego}
Dla celów symulacji ruchu drogowego często wykorzystywane są automaty komórkowe. Zaletą zastosowania automatów komórkowych jest ich oparcie o proste zasady i wysoka wydajność obliczeniowa. Popularny model symulacji ruchu pojazdów opracowali Kai Nagel i Michael Schreckenberg w roku 1992 \cite{nasch}. We wspomnianej pracy opisują oni automat komórkowy którego działanie opiera się na 4 etapach: przyśpieszeniu, hamowaniu, losowości i przesunięciu. Zastosowanie automatów komórkowych w symulacji różnych sytuacji ruchu drogowego opisuje w swojej pracy dyplomowej Maciej Bartodziej \cite{bartodziej}.

Ze względu na swoją prostotę automaty komórkowe można również prosto modyfikować. Przykład symulacji ruchu drogowego w zmiennych warunkach pogodowych przedstawiają w swoim artykule Marcin Bernaś i Bartłomiej Płaczek \cite{bernas+placzek}. Dla celu niniejszej pracy opracowany został symulator stosujący automat komórkowy zmodyfikowany w celu umożliwienia symulacji zespołów skrzyżowań, dokładny opis automatu i jego modyfikacji znajduje się w rozdziale \ref{chap:symulacja} na stronie \pageref{chap:symulacja}.

\section{Istniejące systemy sterowania ruchem}
Inteligentne systemy transportu (ITS\footnote{intelligent transportation systems}) zyskują coraz większą popularność w zarządzaniu ruchem drogowym.
Mają one na celu skrócenie czasu podróży, poprawę bezpieczeństwa czy zwiększenie komfortu podróży co w swojej prezentacji, przedstawiające innowacje w zarządzaniu transportem miejeskim zauważają Aneta Pluta-Zarembska, Marzenna Cichosz i Katarzyna Nowicka ze Szkoły Głównej Handlowej w Warszawie \cite{pluta-zaremba+cichosz+nowicka}. Jako przykład wdrożenia ITS przedstawiony został inteligentny system zarządzania transportem we Wrocławiu. Cel Wrocławskiego systemu, przyspieszenie ruchu o 20\%,  ma zostać osiągnięty, do III kwartału 2014 roku, dzięki zastosowaniu sterowania ruchem na 153 skrzyżowaniach, usprawnieniu komunikacji miejskiej czy dynamicznych informacjach dla kierowców.

Podobny system, zastosowany w trójmieście, prezentują, w artykule na temat systemu TRISTAR, Kazimierz Jamroz i Jacek Oskarbski \cite{jamroz+oskarbski}. Jak zauważają, motywacją wprowadzenia systemu jest między innymi: zatłoczenie infrastruktury drogowej, ryzyko zdarzeń drogowych czy brak informacji o warunkach ruchu. System TRISTAR składa się z połączonych komponentów, takich jak:
\begin{itemize}
	\item zarządzanie ruchem drogowym
	\item zarządzanie transportem zbiorowym
	\item zarządzanie służbami ratowniczymi
	\item zintegrowany system informacji
	\item zarządzanie transportem towarowym
\end{itemize}
Zarządzenie ruchem drogowym obejmuje zarządzanie ruchem miejskim jak i ruchem na okolicznych drogach krajowych i szybkiego ruchu (w postaci trójmiejskiej obwodnicy).

Kolejnym przykładem systemu wdrażanego w Polsce jest podhalański system sterowania ruchem, przedstawiony w artykule Patryka Zakrzewskiego dla magazynu \textit{Drogi Budownictwo Infrastrukturalne} \cite{zakrzewski}. Podobnie jak w przypadku poprzednich dwóch systemów, celami systemu jest zwiększenie bezpieczeństwa i zmniejszenie zatłoczenia na drogach. Jednakże, w przeciwieństwie do nich, podhalański system projektowany był z myślą o zastosowaniu poza warunkami miejskimi. Jego głównym elementem jest monitoring dróg regionu oraz przedstawienie kierowcom informacji o czasie przejazdu do miejscowości orientacyjnych.