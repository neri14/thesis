\chapter{Wstęp}
\section{Wprowadzenie}
wprowadzenie w tematykę pracy
\section{Cel pracy}
cel pracy, motywacja
\section{Układ pracy}
krótki opis co w rozdziałach - możliwe że większość znajduje się w przeglądzie literatury
\section{Przegląd literatury}
\subsection{Wymagania prawne dotyczące sygnalizacji świetlnej}
Podstawowe zasady projektowania sygnalizacji świetlnej oraz jej programów opisuje Rozporządzenie Ministra Infrastruktury z dnia 3 lipca 2003 roku \cite{rozporzadzenie}.
Ze względu na tematykę pracy, najważniejszy jest załącznik nr 3 rozporządzenia opisujący szczegółowe warunki techniczne dla sygnałów drogowych w tym wymagania dotyczące programu sygnalizacji świetlnej w punkcie ósmym.
Wymagania programu sygnalizacji świetlnej zaczynają się od zasad ogólnych, opisujących między innymi programy przejściowe z i do sygnału ostrzegawczego (oznaczającego wyłączenie sygnalizacji). Następnie opisane są wymagania formalne w tym wymagania czasowe dotyczące sygnałów, opisane dokładniej w rozdziale \ref{sec:model_ograniczenia}. na stronie \pageref{sec:model_ograniczenia}.
W dalszej części opisane są wymagania bezpieczeństwa ruchu w tym metoda wyliczenia długości czasów międzyzielonych, czyli odstęp czasu służący zabezpieczeniu aby pojazdy w kolizyjnych strumieniach ruchu nie znalazły się w tym samym miejscu i czasie. Metoda wyliczania czasów międzyzielonych również opisana jest w rozdziale \ref{sec:model_ograniczenia}.

\subsection{Metody stosowane w sterowaniu ruchem drogowym}
Usystematyzowany przegląd metod i algorytmów sterowania ruchem drogowym przedstawiają w swojej pracy Piotr Kawalec z Politechniki Warszawskiej oraz Sylwia Sobieszuk-Durka z Urzędu m. st. Warszawy \cite{kawalec+sobieszuk-durka}. Dzielą oni metody sterowania ruchem drogowym przede wszystkim na optymalizujące i nieoptymalizujące funkcji celu. Metody optymalizujące funkcję celu mogą być przeniesione na teorię systemów sterowania oraz pozwalają na ocenę opracowanego systemu sterowania ruchem drogowym.

Autorzy zauważają, że nawet najlepszy algorytm sterowania ruchem nie da dobrych efektów jeśli zostanie zastosowany dla pojedynczego skrzyżowania. Stąd konieczne jest opracowywanie systemów sterowania z myślą o większych zespołach w których sygnał sterujący wpływa na wiele miejsc w zakresie rozpatrywanego obszaru.

We wspomnianej pracy przedstawione zostały przykładowe struktury systemów adaptacyjnego sterowania ruchem drogowym co jest dobrym punktem wyjścia do projektowania inteligentnego systemu sterowania ruchem. Opis opracowanej struktury sterowania znajduje się w rozdziale \ref{sec:model_opis} na stronie \pageref{sec:model_opis}.

TODO praca \cite{ruchaj}

\subsection{Rozwiązania w symulacji ruchu drogowego}

referencja \ref{chap:symulacja}

TODO praca \cite{nasch}

TODO praca \cite{bernas+placzek}

TODO praca \cite{bartodziej}