\chapter{Wstęp}
\section{Wprowadzenie}
Inteligentne systemy sterowania ruchem drogowym są aktualnym trendem rozwoju miejskiej infrstruktury drogowej w Polsce i stale zyskują na popularności.
Systemy takie tworzone są częściowo w opraciu o istniejącą infrastrukturę i pozwalają na optymalizację przepływu pojazdów w sterowanym obszarze.
Celami stawianymi systemom kontroli ruchu drogowego są: skrócenie czasu podróży, zwiększenie bezpieczeństwa czy poprawa komfortu podróży.

Kontrola ruchu drogowego może być oparta o maksymalizację wykorzystania sieci drogowej jak i minimalizację czasu przejazdów, w zależności od celów stawianych systemowi. Częstymi rozwiązaniami jest poprawa warunków ruchu pojazdów komunikacji zbiorowej, nawet jeśli prowadzi to do zmniejszenia priorytetu ruchu pojazdów indywidualnych.

Rozwiązanie problemu sterowania ruchem może opierać się o systemy uczące się czy systemy wykorzystujące metody systemów sterowania. W poniższej pracy przedstawiona została próba stworzenia systemu sterowania ruchem drogowym w oparciu o matematyczny model ruchu drogowego i wykorzystujący go algorytm sterowania.

\section{Cel pracy}
Celem niniejszej pracy jest próba opracowania systemu sterowania ruchem drogowym na obszarze wzorowanym na okolicach placu Grunwaldzkiego we Wrocławiu. Opracowany algorytm sterowania zostanie porównany z ruchem drogowym w przypadku braku sterowania jak i w przy zastosowaniu stałego, niezsynchronizowanego programu stałoczasowej sygnalizacji świetlnej.

Motywacją w powstaniu tej pracy jest próba stworzenia systemu który będzie dynamicznie reagował na zmienną charakterystykę ruchu drogowego oraz który będzie stosował się do zadanych, wymaganych przez prawo ograniczeń.