\chapter{Wstęp}
\section{Wprowadzenie}
Inteligentne systemy sterowania ruchem drogowym są aktualnym trendem rozwoju miejskiej infrstruktury drogowej w Polsce i stale zyskują na popularności.
Systemy takie tworzone są częściowo w opraciu o istniejącą infrastrukturę i pozwalają na optymalizację przepływu pojazdów w sterowanym obszarze.
Celami stawianymi systemom kontroli ruchu drogowego są: skrócenie czasu podróży, zwiększenie bezpieczeństwa czy poprawa komfortu podróży.

Kontrola ruchu drogowego może być oparta, w zależności od celów stawianych systemowi, o maksymalizację wykorzystania sieci drogowej jak i minimalizację czasu przejazdów. Częstym celem, zastosowania inteligentnych systemów sterowania, jest poprawa warunków ruchu pojazdów komunikacji zbiorowej, nawet jeśli prowadzi to do zmniejszenia priorytetu ruchu pojazdów indywidualnych.

Inteligentny system sterowania ruchem może opierać się o systemy uczące się czy systemy wykorzystujące metody optymalizacyjne. W poniższej pracy przedstawiona została próba stworzenia adaptacyjnego systemu sterowania ruchem drogowym w oparciu o algorytm sterowania wykorzystujący matematyczny model ruchu drogowego.

\section{Cel pracy}
Celem niniejszej pracy jest próba opracowania inteligentnego, adaptacyjnego, systemu sterowania ruchem drogowym na obszarze wzorowanym na okolicach placu Grunwaldzkiego we Wrocławiu. Przygotowany algorytm sterowania zostanie porównany z zachowaniem ruchu drogowego w przypadku braku sterowania jak i przy zastosowaniu stałego, niezsynchronizowanego, programu, stałoczasowej sygnalizacji świetlnej.

Motywacją w powstaniu tej pracy jest próba stworzenia systemu który będzie dynamicznie reagował na zmienną charakterystykę ruchu drogowego oraz będzie stosował się do, wymaganych przez prawo, ograniczeń.