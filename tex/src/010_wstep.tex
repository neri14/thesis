\chapter{Wstęp}
\section{Wprowadzenie}
Inteligentne systemy sterowania ruchem drogowym są aktualnym trendem rozwoju miejskiej infrstruktury drogowej w Polsce i stale zyskują na popularności.
Systemy takie tworzone są częściowo w opraciu o istniejącą infrastrukturę i pozwalają na optymalizację przepływu pojazdów w sterowanym obszarze.
Celem zastosowania systemów kontroli ruchu drogowego może być: skrócenie czasu podróży, zwiększenie bezpieczeństwa czy poprawa komfortu podróży. Często priorytetem jest również poprawa działania komunikacji zbiorowej, nawet jeśli prowadzi to do pogorszenia warunków ruchu komunikacji indywidualnej.
W związku z różnorodnością celów stawianych systemowi kontroli ruchu, może być on oparty o maksymalizacje wykorzystania sieci drogowej jak i minimalizację czasu przejazdu pojazdów.

Inteligentny system sterowania ruchem może opierać się o systemy uczące się jak i systemy wykorzystujące metody optymalizacyjne. W poniższej pracy przedstawiona została próba stworzenia adaptacyjnego systemu sterowania ruchem drogowym w oparciu o algorytm sterowania wykorzystujący matematyczny model ruchu drogowego.

\section{Cel pracy}
Celem niniejszej pracy jest próba opracowania inteligentnego, adaptacyjnego, systemu sterowania ruchem drogowym na obszarze wzorowanym na okolicach placu Grunwaldzkiego we Wrocławiu. Przygotowany algorytm sterowania zostanie porównany z niezsynchronizowanym, niezmiennym, stałoczasowym programem sygnalizacji świetlnej oraz brakiem sterowania.

Motywacją w powstaniu tej pracy jest próba stworzenia systemu który będzie dynamicznie reagował na zmienną charakterystykę ruchu drogowego jednocześnie stosując się do, wymaganych przez prawo, ograniczeń.