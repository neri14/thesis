\chapter{Algorytm sterowania ruchem drogowym}
Do wyznaczenia nowego stanu sygnalizatora używamy programowania dynamicznego przyjmując jako stan początkowy aktualny stan systemu sterowania. Ponieważ zakładamy, że zarówno zmienna stanu jak i sterowanie ma postać dyskretną, przygotowujemy tabele stanów które zawierają wszystkie możliwe stany dla, określonego parametrem, czasu w przód. Jako nowe sterowanie sygnalizatorów przyjmujemy takie które da najlepszy wynik.

\section{Programowanie dynamiczne}
Na podstawie przykładów i zadań przygotowanych przez Katarzynę Jakowską-Suwalską \cite{jakowska-suwalska}.
Programowanie dynamiczne polega na wieloetapowym rozwiązywaniu problemów decyzyjnych. Rozwiązaniem problemu programowania dynamicznego jest ciąg decyzji, który jest strategią optymalną, dla której funkcja oceny rozważanego problemu przyjmuje wartość ekstremalną.

Programowanie dynamiczne można zastosować do problemu którego stan końcowy zależy od stanów poprzedzających oraz opracowanej strategii. W przypadku sterowania ruchem drogowym wyznaczamy pewien przedział czasu dla którego chcemy wyznaczyć stan sygnalizacji. Chcemy również minimalizować funkcję celu (\ref{eq:f_celu}).

W przypadku dyskretnego programowania dynamicznego wyznaczamy wszystkie możliwe stany na podstawie możliwych decyzji i wybieramy decyzje optymalne. Dla rozpatrywanego przykładu, zbiór możliwych decyzji składa się z prawidłowych stanów sygnalizatorów (takich które nie spowodują jednoczesnego zezwolenia ruchu strumieni kolizyjnych).

\subsection{Parametry modelu}
b -- przepływ pojazdów po nadaniu sygnału zezwalającego na przejazd

c -- długość cyklu świetlnego

p -- wpływ wielkości kolejek na sterowanie

T -- długość odcinka czasu w przód na podstawie którego wyznaczamy stan sygnalizacji

\begin{equation}
	\begin{array}{c}
		b \in \mathbb{N}\\
		c \in \mathbb{N}\\
		p \in <0.0, 1.0>\\
		T \in \mathbb{N}
	\end{array}
\end{equation}

\subsection{Model w przestrzeni stanu}
Stan obiektu sterowania definiujemy jako:

\begin{equation}
	\begin{array}{c}
		X^{(1)} (n) = [x^{(1)}_{i} (n)]_{i \in <1,s>}\\
		x^{(1)}_{i} (n) \in \mathbb{N}
	\end{array}
\end{equation}

x^{(1)}_{i} (n) \textrm{ -- wielkość kolejki przed i-tym sygnalizatorem w n-tej chwili czasu}

s -- liczba sygnalizatorów

\begin{equation}
	\begin{array}{c}
		X^{(2)} (n) = [x^{(2)}_{i} (n)]_{i \in <1,s>}\\
		x^{(2)}_{i} (n) \in \mathbb{N}
	\end{array}
\end{equation}

x^{(2)}_{i} (n) \textrm{ -- czas trwania aktualnego sygnału na i-tym sygnalizatorze w n-tej chwili czasu}

\vspace{1.5cm}
Dany stan początkowy:

\begin{equation}
	\begin{array}{c}
		X^{(1)} (0) = K\\
		X^{(2)} (0) = C
	\end{array}
\end{equation}

K -- wielkości kolejek w chwili czasu dla jakiej wyznaczamy stan sygnalizatorów

C -- czas nadawania aktualnych sygnałów w chwili czasu dla jakiej wyznaczamy stan sygnalizatorów

\vspace{1.5cm}
Zakłócenia (przepływ pojazdów):

\begin{equation}
	\begin{array}{c}
		Z (n) = [z_{i} (n)]_{i \in <1,s>}\\
		z_{i} (n) \in \mathbb{N}
	\end{array}
\end{equation}

z_{i} (n) \textrm{ -- przepływ pojazdów w kierunku i-tego sygnalizatora w chwili n}

\begin{equation}
	Z (0) = P
\end{equation}

P -- wielkości przepływów w chwili czasu dla jakiej wyznaczamy stan sygnalizatorów

\vspace{1.5cm}
Dla chwili czasu większej od zera przyjmujemy, że wielkość przepływu równa jest aktualnemu przepływowi jeśli strumień pojazdów nie jest sterowany przez sąsiedni kontroler. W przeciwnym wypadku wielkość przepływu równa jest przewidywanemu przepływowi udostępnianemu przez sąsiedni kontroler.

\vspace{1.5cm}
Wielkość wyjściowa (stany sygnalizatorów):
\begin{equation}
	\begin{array}{c}
		U (n) = [u_{i} (n)]_{i \in <1,s>}\\
		u_{i} \in {0, 1}
	\end{array}
\end{equation}

u_{i} (n) \textrm{ -- stan i-tego sygnalizatora w chwili n, 1 - zezwolenie na wjazd, 0 - brak zezwolenia}

\vspace{1.5cm}
Równanie stanu:

\begin{equation}
	x^{(1)}_{i} (n+1) = x^{(1)}_{i} (n) + z_{i} (n) - u_{i} (n) \cdot b
\end{equation}

\begin{equation}
	\begin{array}{c}
		x^{(2)}_{i} (n+1) = 0 \textrm{ dla } u_{i} (n+1) \neq u_{i} (n)\\
		x^{(2)}_{i} (n+1) = x^{(2)}_{i} (n) + 1 \textrm{ dla } u_{i} (n+1) = u_{i} (n)\\
	\end{array}
\end{equation}

\vspace{1.5cm}
Funkcja celu:

\begin{equation}
	\label{eq:f_celu}
	g(X^{(1)} (n), X^{(2)} (n), U (n)) =
		p \cdot \sum\limits_{i=1}^{s} \frac{x^{(1)}_{i} (n)}{l_{i}}
		+ (1-p) \cdot \sum\limits_{i=1}^{s} \frac{(1-u_{i} (n)) \cdot x^{(2)}_{i} (n)}{c}
\end{equation}

l_{i} \textrm{ -- maksymalna mierzona wielkość kolejki przed i-tym sygnalizatorem}

\section{Uwzględnienie ograniczeń cyklów świetlnych}
Podstawowym ograniczeniem, spośród wymienionych w rozdziale \label{sec:model_ograniczenia}, jakiego nie uwzględnia tak zdefiniowany problem programowania dynamicznego jest ograniczenie faz cyklu świetlnego. Jak już zostało wspomniane sygnały mogą być nadawane jedynie w sekwencji czerwony, czerwono-żółty, zielony, żółty, czerwony. Natomiast powyższy model podejmuje jedynie decyzję czy wydać zgodę na ruch czy nie.

Za przestrzeganie tego jak i innych, czasowych, ograniczeń odpowiada komponent który przekazuje do obiektu sterowanego ostateczną decyzję.

Zmiana sygnału powoduje utworzenie sekwencji sygnałów które nastąpią w kolejnych chwilach czasu, dodatkowo uruchomienie sygnału zielonego jest odpowiednio opóźnione przez uwzględnienie czasów międzyzielonych. Długości sygnałów żółtego i czerwono-żółtego są wprowadzane razem ze wspomnianą sekwencją. Również minimalna długość sygnału zielonego jest w ten sposób przedstawiana.

Kolejnym elementem pracy wspomnianego komponentu jest wymuszenie uruchomienia sygnału czerwonego przynajmniej raz na cykl świetlny.

\section{Metody oceny skuteczności algorytmu}
Do oceny skuteczności algorytmu wykorzystać można określone na podstawie pracy Piotra Kawalca i Sylwii Sobieszuk-Durki \cite{kawalec+sobieszuk-durka} metody oceny. Są to:
\begin{itemize}
	\item średnie czasy przejazdu pojazdów na wybranych kierunkach
	\item liczba zatrzymań pojazdów
	\item średnia prędkość pojazdów
\end{itemize}

W ten sposób określone metody oceny można wykorzystać do porównania algorytmu z przypadkiem braku sterowania czy sterowaniem przy użyciu prostego, stałoczasowego, niezsynchronizowanego, programu sygnalizacji. Ocena wyników badań znajduje się w rozdziale \ref{chap:ocena}.