\section{Algorytm sterowania ruchem drogowym}
\begin{figure}[h]
    \centering
    \includegraphics[width=0.5\textwidth]{images/kontroler.pdf}
    \caption{Model sterowania kontrolera}
    \label{fig:kontroler}
\end{figure}

Jak widać na rysunku \ref{fig:model} i \ref{fig:kontroler}, kontroler, do wyznaczenia stanu sygnalizatorów wykorzystuje, stan obiektu sterowanego. Stan obiektu składa się z trzech zmiennych.

\vspace{1.5cm}
\textbf{Parametry algorytmu}

p \in <0.0, 1.0> \textrm{ -- wpływ przewidywanego przepływu pojazdów na sterowanie}

q \in <0.0, 1.0> \textrm{ -- wpływ aktualnego przepływu pojazdów na sterowanie}

r \in <0.0, 1.0> \textrm{ -- wpływ aktualnych wielkości kolejek na sterowanie}

\vspace{1.5cm}
\textbf{Stałe algorytmu}

s \in \mathbb{N} \textrm{ -- liczba sygnalizatorów w kontrolowanym obszarze}

m \in \mathbb{N} \textrm{ -- maksymalny możliwy do zmierzenia przepływ pojazdów}

\begin{equation}
	\begin{array}{c}
		K = \left[ k_{i} \right]_{i \in <0,s>}\\
		k_{i} \in \mathbb{N}
	\end{array}
\end{equation}

k_{i} \textrm{ -- maksymalna możliwa wielkość kolejki przed i-tym sygnalizatorem}

\vspace{1.5cm}
\textbf{Zmienne stanu}

\begin{equation}
	\begin{array}{c}
		X^{(1)} (n) = \left[ x^{(1)}_{i} (n) \right]_{i \in <0,s>}\\
		x^{(1)}_{i} (n) \in \mathbb{N}
	\end{array}
\end{equation}

x^{(1)}_{i} (n) \textrm{ -- przewidywany przyszły przepływ pojazdów w kierunku i-tego sygnalizatora}

Przewidywany przepływ pojazdów jest wyznaczany na podstawie stanu sygnalizatorów sąsiadujacych kontrolerów.

\begin{equation}
	\begin{array}{c}
		X^{(2)} (n) = \left[ x^{(2)}_{i} (n) \right]_{i \in <0,s>}\\
		x^{(2)}_{i} (n) \in \mathbb{N}
	\end{array}
\end{equation}

x^{(2)}_{i} (n) \textrm{ -- przepływ pojazdów w kierunku i-tego sygnalizatora w chwili n}

\begin{equation}
	\begin{array}{c}
		X^{(3)} (n) = \left[ x^{(3)}_{i} (n) \right]_{i \in <0,s>}\\
		x^{(3)}_{i} (n) \in \mathbb{N}
	\end{array}
\end{equation}

x^{(3)}_{i} (n) \textrm{ -- wielkość kolejki przed i-tym sygnalizatorem w chwili n}

\vspace{1.5cm}
\textbf{Algorytm}

Proponowany jest poniższy algorytm wyznaczania nowego stanu sygnalizatorów.
\begin{enumerate}
	\item wyznaczenie wszystkich, zbiorów bezkolizyjnych stanów sygnalizatorów
	\item wyliczenie wag sygnalizatorów
	\item wyliczenie funkcji oceny jako sumy wag sygnalizatorów zezwalających na wjazd dla wyznaczonych wcześniej zbiorów
	\item wyznaczenie optymalnego, o najwyższej wartości funkcji oceny, stanu
\end{enumerate}

\vspace{0.5cm}
W pierwszym kroku algorytmu wyznaczane są wszystkie możliwe zbiory bezkolizyjnych stanów sygnalizatorów.
Spośród tak wyznaczonych zbiorów możliwy jest późniejszy wybór takiego który będzie w danym momencie optymalny.

\vspace{0.5cm}
Kolejny krok pozwala wyznaczyć wagi zezwolnia na ruch przez każdy z sygnalizatorów.

\begin{equation}
	w_{i} (n) = p \cdot \frac{x^{(1)}_{i} (n)}{m} + q \cdot \frac{x^{(2)}_{i} (n)}{m} + r \cdot \frac{x^{(3)}_{i} (n)}{k_{i}}
\end{equation}

w_{i} (n) \textrm{ -- waga i-tego sygnalizatora w chwili n}

\vspace{0.5cm}
W następnym kroku, wyliczone wagi są sumowane aby wyznaczyć wartość funkcji oceny dla każdego, przygotowanego w pierwszym kroku, zbioru.

\begin{equation}
	Q (X(n), S') = \sum\limit_{i \in S'} w_{i}
\end{equation}

S' \textrm{ -- wyznaczony zbiór sygnalizatorów}

\vspace{0.5cm}
Na końcu wybierany jest stan optymalny, o najwyższej funkcji oceny.

\section{Uwzględnienie ograniczeń cyklów świetlnych}
Podstawowym ograniczeniem, spośród wymienionych w rozdziale \label{sec:model_ograniczenia}, jakiego nie uwzględnia tak zdefiniowany algorytm jest ograniczenie faz cyklu świetlnego. Jak już zostało wspomniane  sygnały mogą być nadawane jedynie w zadanej sekwencji. Natomiast powyższy model podejmuje jedynie decyzję o wydaniu zgody na ruch.

Za przestrzeganie tego jak i innych, czasowych, ograniczeń odpowiada dodatkowy komponent który przekazuje do obiektu sterowanego ostateczną decyzję. Zmiana sygnału powoduje utworzenie, przez ten komponent, sekwencji sygnałów które nastąpią w kolejnych chwilach czasu, dodatkowo uruchomienie sygnału zielonego jest odpowiednio opóźnione przez uwzględnienie czasów międzyzielonych. Również ograniczenia długości sygnałów żółtego i czerwono-żółtego są uwzględniane na tym etampie. Wspomniany komponent odpowiada również za wymuszenie uruchomienia sygnału czerwonego przynajmniej raz na cykl świetlny.

\section{Metody oceny skuteczności algorytmu}
Do oceny skuteczności algorytmu wykorzystać można określone na podstawie pracy Piotra Kawalca i Sylwii Sobieszuk-Durki \cite{kawalec+sobieszuk-durka} metody oceny. Są to:
\begin{itemize}
	\item średnie czasy przejazdu pojazdów na wybranych kierunkach
	\item liczba zatrzymań pojazdów
	\item średnia prędkość pojazdów
\end{itemize}

W ten sposób określone metody oceny można wykorzystać do porównania algorytmu z przypadkiem braku sterowania czy sterowaniem przy użyciu prostego, stałoczasowego, niezsynchronizowanego, programu sygnalizacji. Ocena wyników badań znajduje się w rozdziale \ref{chap:ocena}.