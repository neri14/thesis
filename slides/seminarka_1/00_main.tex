% -*- TeX:PL -*-
% $Id: $
\documentclass[17pt]{beamer}
\usepackage[T1]{polski}
\usepackage[utf8]{inputenc}
\usepackage{lipsum}
\usepackage{multimedia}

\author{Przemysław Rokosz}
\title{Systemy CDN}
 \subtitle{metody, algorytmy}
 \date{}
 \institute{Wydział Informatyki i Zarządzania\\\vspace{\baselineskip}Seminarium - Systemy Webowe\\cz 10:15 - 13:00}
 \subject{Systemy CDN metody algorytmy}
 \keywords{Systemy CDB, metody, algorytmy}
 \titlegraphic{}

\DeclareGraphicsExtensions{.pdf,.png,.jpg}

\setbeamersize{text margin left=0mm,text margin right=2.5mm}
\usetheme[]{pwr}

\setbeamertemplate{footline}
{
\leavevmode
\hbox{
\begin{beamercolorbox}[wd=.88\paperwidth,ht=2.5ex,dp=1.125ex,right]{title in head/foot}
\end{beamercolorbox}
\begin{beamercolorbox}[wd=.1\paperwidth,ht=2.5ex,dp=1.125ex,center]{title in head/foot}
\usebeamerfont{author in head/foot} \insertframenumber/\inserttotalframenumber
\end{beamercolorbox}
}
\vskip0pt%
}

\setbeamertemplate{navigation symbols}{}

\hypersetup{
 urlcolor=blue
}

\begin{document}
\begin{frame}[plain,t]
 \maketitle
\end{frame}

\begin{frame}[shrink=5]
 \frametitle{\vspace{22px}Plan prezentacji}
 \begin{itemize}
  \item{Co to jest CDN?}
  \item{Sposoby dostarczania treści}
  \item{Współpraca serwerów krawędziowych}
  \item{Algorytmy rozmieszczenia treści}
  \item{Metody globalnego kierowania żądań}
  \item{Wybór najlepszego serwera dla żądania}
  \item{Podsumowanie}
  \item{Bibliografia}
 \end{itemize}
\end{frame}

\end{document}
