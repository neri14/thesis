% -*- TeX:PL -*-
% $Id: $
\documentclass[17pt]{beamer}
\usepackage[T1]{polski}
\usepackage[utf8]{inputenc}
\usepackage{lipsum}
\usepackage{multimedia}

\author{}
\title{Opracowanie inteligentnego systemu sterowania ruchem drogowym}
 \subtitle{}
 \date{}
 \institute{autor: inż. Przemysław Rokosz\\\vspace{\baselineskip}promotor: dr inż. Grzegorz Filcek}
 \subject{Opracowanie inteligentnego systemu sterowania ruchem drogowym}
 \keywords{Inteligentny System Sterowania Ruchem Drogowym}
 \titlegraphic{}

\DeclareGraphicsExtensions{.pdf,.png,.jpg}

\setbeamersize{text margin left=0mm,text margin right=2.5mm}
\usetheme[]{pwr}

\setbeamertemplate{footline}
{
\leavevmode
\hbox{
\begin{beamercolorbox}[wd=.88\paperwidth,ht=2.5ex,dp=1.125ex,right]{title in head/foot}
\end{beamercolorbox}
\begin{beamercolorbox}[wd=.1\paperwidth,ht=2.5ex,dp=1.125ex,center]{title in head/foot}
\usebeamerfont{author in head/foot} \insertframenumber/\inserttotalframenumber
\end{beamercolorbox}
}
\vskip0pt%
}

\setbeamertemplate{navigation symbols}{}

\hypersetup{
 urlcolor=blue
}

\begin{document}
\begin{frame}[plain,t]
 \maketitle
\end{frame}

\begin{frame}[shrink=5]
 \frametitle{\vspace{22px}Plan prezentacji}
 \begin{itemize}
  \item{Sterowanie ruchem drogowym}
  \item{Dotychczasowe rozwiązania}
  \item{Cel pracy - opis systemu}
  \item{Stan prac}
  \item{Bibliografia}
 \end{itemize}
\end{frame}

\begin{frame}[shrink=5]
 \frametitle{\vspace{22px}Sterowanie ruchem drogowym}
 \begin{itemize}
  \item{Na czym polega sterowanie ruchem drogowym?}
  \item{Jakie są wyzwania sterowania ruchem?}
  \item{Jakie są obostrzenia prawne?}
 \end{itemize}
\end{frame}

\begin{frame}[shrink=5]
 \frametitle{\vspace{22px}Dotychczasowe rozwiązania}
 \begin{itemize}
  \item{Kilka przykładów rozwiązań, w tym wrocławskie ITS}
 \end{itemize}
\end{frame}

\begin{frame}[shrink=5]
 \frametitle{\vspace{22px}Cel pracy - opis systemu}
 \begin{itemize}
  \item{Definicja celu pracy}
  \item{Jakie badania - (porównanie z prostym systemem sterowania (stały program), brak sterowania)}
  \item{m. in. Badanie zachowania (przygotowanych wersji) dla nagłych zmian ruchu}
  \item{Ogólna charakterystyka systemu}
  \item{Inteligentne sterowanie ruchem drogowym z punktu widzenia systemów sterowania}
 \end{itemize}
\end{frame}

\begin{frame}[shrink=5]
 \frametitle{\vspace{22px}Stan prac}
 \begin{itemize}
  \item{Komunikacja między elementami systemu - gotowe}
  \item{Symulator - wczytywanie symulowanego świata}
 \end{itemize}
\end{frame}

\begin{frame}[shrink=5]
 \frametitle{\vspace{22px}Bibliografia}
 \begin{itemize}
  \item{Na podstawie przygotowanego podsumowania}
 \end{itemize}
\end{frame}

\end{document}
